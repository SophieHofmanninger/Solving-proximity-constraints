%%%%%%%%%%%%%%%%%%%%%%%%%%%%%%%%%%%%%%%%%
% Lachaise Assignment
% LaTeX Template
% Version 1.0 (26/6/2018)
%
% This template originates from:
% http://www.LaTeXTemplates.com
%
% Authors:
% Marion Lachaise & François Févotte
% Vel (vel@LaTeXTemplates.com)
%
% License:
% CC BY-NC-SA 3.0 (http://creativecommons.org/licenses/by-nc-sa/3.0/)
% 
%%%%%%%%%%%%%%%%%%%%%%%%%%%%%%%%%%%%%%%%%

%----------------------------------------------------------------------------------------
%	PACKAGES AND OTHER DOCUMENT CONFIGURATIONS
%----------------------------------------------------------------------------------------

\documentclass{article}

\input{structure.tex} % Include the file specifying the document structure and custom commands

%----------------------------------------------------------------------------------------
%	ASSIGNMENT INFORMATION
%----------------------------------------------------------------------------------------

\title{Project: Solving proximity constraints} % Title of the assignment

\author{Jan-Michael Holzinger\thanks{jan.holzinger@gmx.at} \and Sophie Hofmanninger\thanks{sophie@hofmanninger.co.at}} % Author name and email address

\date{JKU Linz --- SS2019} % University, school and/or department name(s) and a date

%----------------------------------------------------------------------------------------

\begin{document}

\maketitle % Print the title

%----------------------------------------------------------------------------------------
%	INTRODUCTION
%----------------------------------------------------------------------------------------

\begin{table}%
\begin{tabular}{[|l|l|l]}
Version Number & Changes 
\end{tabular}
\caption{}
\label{}
\end{table}


\section{System Overview}

We split the problem in 4 (5) smaller tasks:
\begin{enumerate}
	\item Input Processing
	\item Pre-Unification
	\item Neighborhood Constraints
	\item Output
	\item [O] Web-Integration
\end{enumerate}





%----------------------------------------------------------------------------------------
%	PROBLEM 1
%----------------------------------------------------------------------------------------

\subsection{Input Processing} 



%------------------------------------------------

\subsection{Theoretical viewpoint}


	
%------------------------------------------------

\subsection{Algorithmic issues}


%----------------------------------------------------------------------------------------
%	PROBLEM 2
%----------------------------------------------------------------------------------------

\section{Implementation}


% File contents
%\begin{file}[hello.py]
%\begin{lstlisting}[language=Python]
%#! /usr/bin/python
%
%import sys
%sys.stdout.write("Hello World!\n")
%\end{lstlisting}
%\end{file}




% Command-line "screenshot"
%\begin{commandline}
	%\begin{verbatim}
		%$ chmod +x hello.py
		%$ ./hello.py
%
		%Hello World!
	%\end{verbatim}
%\end{commandline}

% Numbered question, with an optional title

%\begin{question}[\itshape (with optional title)]
	%In congue risus leo, in gravida enim viverra id.
%\end{question}

%\begin{question}
	%Quisque ullamcorper placerat ipsum. Cras nibh. Morbi vel justo vitae lacus tincidunt ultrices. Lorem ipsum dolor sit amet, consectetuer adipiscing elit.
%
	%% Subquestions numbered with letters
	%\begin{enumerate}[(a)]
		%\item Do this.
		%\item Do that.
		%\item Do something else.
	%\end{enumerate}
%\end{question}



% Warning text, with a custom title
%\begin{warn}[Notice:]
  %In congue risus leo, in gravida enim viverra id.
%\end{warn}
%\begin{info} % Information block
	%This is an interesting piece of information, to which the reader should pay special attention. 
%\end{info}

% Math equation/formula
%\begin{equation}
%	I = \int_{a}^{b} f(x) \; \text{d}x.
%\end{equation}

%----------------------------------------------------------------------------------------

\end{document}
