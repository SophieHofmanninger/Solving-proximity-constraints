\section{Introduction} \subsection{}

\begin{frame}[fragile=singleslide]
	\frametitle{Motivation}
		For proving theorems, a frequently occurring problem is to find common instances of formulae.
		\begin{exampleblock}{Example 1}
			Let \(f\) be a function, \(a,b\) constants and \(x\) a variable. The two expressions
			\[f(a,x)\quad\text{and}\quad f(a,b)\]
			can be unified with \(\{x\mapsto b\}\).
		\end{exampleblock}

  \end{frame}	
		
%---------------------------------------------------------------

\begin{frame}[fragile=singleslide]
	\frametitle{Motivation}
		For proving theorems, a frequently occurring problem is to find common instances of formulae.
		\begin{exampleblock}{Example 2}
			Let \(f,g\) be functions, \(a,b\) constants and \(x\) a variable. The two expressions
			\[f(a,x)\quad\text{and}\quad g(a,b)\]
			cannot be unified as \(f\neq g\).
		\end{exampleblock}

  \end{frame}	
		
%---------------------------------------------------------------
				
\begin{frame}[fragile=singleslide]
	\frametitle{Motivation}
		In 1965 Robinson presented his unification algorithm and solved this problem, his algorithm was improved for better(=faster) performance since.\\ \ \\
		If we consider now the unification problem \[f(a,x)\simeq^{?} g(a,b)\] again, we might wonder, if we could not ignore \(f\neq g\), if they are ``close'' to each other, i.e. if they are equal in a fuzzy logic sense.\\
		Being close is represented as a proximity relation, which are symmetric and reflexive, but not necessarily transitive. C. Pau and T. Kutsia solved this problem, presenting an algorithm, which we implemented.

  \end{frame}	
		
%---------------------------------------------------------------


	\begin{frame}[fragile=singleslide]
	\frametitle{Introduction}
		4 sets:
		\begin{itemize}
			\item P: unification problem to be solved
			\item C: neighborhood constraint
			\item $\sigma$: set of pre-unifier
			\item $\Phi$: name-class mapping
		\end{itemize}
  \end{frame}	
		
%---------------------------------------------------------------
		
	\begin{frame}[fragile=singleslide]
	\frametitle{Pre-Unification rules}
		(Tri) Trivial: \ldots \\
		(Dec) Decomposition: \ldots \\
		\ldots
	
  \end{frame}	
		
%---------------------------------------------------------------
		
		\begin{frame}[fragile=singleslide]
	\frametitle{Rules for Neigborhood Constraints}
		\textcolor[rgb]{0.55,0,0}{(FFS)} $\{f \approx^? g\} \uplus C; \Phi \Rightarrow  C; \Phi; \text{ if } \mathcal{R}(f,g)\geq \lambda$ \\
		\vspace{0.3cm}
		\textcolor[rgb]{0.55,0,0}{(NFS)} $\{N \approx^? g\} \uplus C; \Phi \Rightarrow  C; update(\Phi,N\rightarrow \textbf{pc}(g,\mathcal{R},\lambda))$  \\
		\vspace{0.3cm}
		\textcolor[rgb]{0.55,0,0}{(FSN)} $\{g \approx^? N\} \uplus C; \Phi \Rightarrow  \{N \approx^? g\} \cup C;\Phi$  \\
		\vspace{0.3cm}
		\textcolor[rgb]{0.55,0,0}{(NN1)} $\{N \approx^? M\} \uplus C; \Phi \Rightarrow  C; update(\Phi,N\rightarrow {f}, M\rightarrow \textbf{pc}(f,\mathcal{R},\lambda))$, where $N \in dom(\Phi )$, $f\in \Phi(N)$ \\
		\vspace{0.3cm}
		\textcolor[rgb]{0.55,0,0}{(NN2)} $\{M \approx^? N\} \uplus C; \Phi \Rightarrow  \{N \approx^? M\} \cup C;\Phi$, where $M \notin dom(\Phi )$, $N\in dom(\Phi)$ \\
		\vspace{0.3cm}
		\textcolor[rgb]{0.55,0,0}{(Fail1)} $\{f \approx^? g\} \uplus C; \Phi \Rightarrow  \perp$, if $\mathcal{R}(f,g) < \lambda$ \\
		\vspace{0.3cm}
		\textcolor[rgb]{0.55,0,0}{(Fail2)} $C; \Phi \Rightarrow  \perp$, if there exists $N \in dom(\Phi )$ such that $\Phi (N)=\emptyset$ \\
		\vspace{0.3cm}
	
  \end{frame}	
	
%---------------------------------------------------------------
		
		\begin{frame}[fragile=singleslide]
	\frametitle{Simple example}
	Pre - fail:  $p(x,a) =? q(f(a),g(b))$\\
		\vspace{0.5cm}
		
		%Pre - ok, CS - fail: \textcolor[rgb]{1,0.41,0.13}{- nur als Beispiel wo deiner funktioniert, aber meiner ned -} $p(x,a) =? q(f(a),b) ; R=\{(b,c),(c,d),(f,g),(p,q)\}$\\
		%\vspace{0.5cm}
		
		\textbf{Test 3 in Test.java}\\
		Pre - ok, CS - fail: \textcolor[rgb]{0,0.58,0}{- des ging bei mir, i hätt 5 Schritte und 2 Branches, die aber failen-}$p(x,y,x) =? q(f(a),g(b),y) ; R=\{(b,c),(c,d),(f,g),(p,q)\}$\\
		\vspace{0.5cm}
		
		\textbf{Test 4 in Test.java}\\
		Pre - ok, CS - ok: \textcolor[rgb]{0,0.58,0}{- des ging bei mir, i hätt 4 Schritte und 1 Branch-}$p(x,z) =? q(f(b),f(x)) ; R=\{(a,a'),(a',b),(b,c'),(c',c),(p,q)\}$\\
		\vspace{0.5cm}
		
		Solution: \ldots
	
  \end{frame}	
	
