\section{Introduction} \subsection{}

\begin{frame}[fragile=singleslide]
	\frametitle{Motivation}
		For proving theorems, a frequently occurring problem is to find common instances of formulae.
		\begin{exampleblock}{Example 1}
			Let \(f\) be a function, \(a,b\) constants and \(x\) a variable. The two expressions
			\[f(a,x)\quad\text{and}\quad f(a,b)\]
			can be unified with \(\{x\mapsto b\}\).
		\end{exampleblock}

  \end{frame}	
		
%---------------------------------------------------------------

\begin{frame}[fragile=singleslide]
	\frametitle{Motivation}
		For proving theorems, a frequently occurring problem is to find common instances of formulae.
		\begin{exampleblock}{Example 2}
			Let \(f,g\) be functions, \(a,b\) constants and \(x\) a variable. The two expressions
			\[f(a,x)\quad\text{and}\quad g(a,b)\]
			cannot be unified as \(f\neq g\).
		\end{exampleblock}

  \end{frame}	
		
%---------------------------------------------------------------
				
\begin{frame}[fragile=singleslide]
	\frametitle{Motivation}
		In 1965 Robinson presented his unification algorithm and solved this problem, his algorithm was improved for better(=faster) performance since.\\ \ \\
		If we consider now the unification problem \[f(a,x)\simeq^{?} g(a,b)\] again, we might wonder, if we could not ignore \(f\neq g\), if they are ``close'' to each other, i.e. if they are equal in a fuzzy logic sense.\\
		Being close is represented as a proximity relation, which are symmetric and reflexive, but not necessarily transitive. C. Pau and T. Kutsia solved this problem, presenting an algorithm, which we implemented.

  \end{frame}	
		
%---------------------------------------------------------------


	\begin{frame}[fragile=singleslide]
	\frametitle{Introduction}
		The Algorithm consists of two sub-algorithms and works on (modifies) 4 sets:
		\begin{itemize}
			\item P: unification problem to be solved ,
			\item C: neighbourhood constraint,
			\item $\sigma$: set of pre-unifier,
			\item $\Phi$: name-class mapping,
		\end{itemize}
		where Algorithm 1 modifies \(P,C,\) and \(\sigma\) and Algorithm 2 modifies \(C\) and \(\Phi\). If Algorithm 1 was successful, \(P=\emptyset\), if Algorithm 2 was successful \(C=\emptyset\). 
  \end{frame}	
		
%---------------------------------------------------------------
		
	\begin{frame}[fragile=singleslide]
	\frametitle{Pre-Unification rules}
	\textcolor[rgb]{0.55,0,0}{(Tri)} \(\{x \simeq^? x\} \uplus P;C;\sigma \Rightarrow  P; C;\sigma\) \\
		\vspace{0.3cm}
		\textcolor[rgb]{0.55,0,0}{(Dec)} \(\{F(\overline{s_n}) \simeq^? G(\overline{t_n})\} \uplus  P;C;\sigma \Rightarrow \{\overline{s_n \simeq^?t_n}\}\cup P;\{F\approx^? G\}\cup  C;\sigma\) \\
		\vspace{0.3cm}
		\textcolor[rgb]{0.55,0,0}{(VE)} \(\{x \simeq^? t\} \uplus P;C;\sigma \Rightarrow  \{t'\simeq^? t\}\cup P{x\mapsto t'}; C;\sigma\{x\mapsto t'\}\) \\
		\vspace{0.3cm}
		\textcolor[rgb]{0.55,0,0}{(Ori)} \(\{t \simeq^? x\} \uplus P;C;\sigma \Rightarrow  \{x \simeq^? t\}\cup P; C;\sigma\) \\
		\vspace{0.3cm}
		\textcolor[rgb]{0.55,0,0}{(Cla)} \(\{F(\overline{s_n}) \simeq^? G(\overline{t_n})\} \uplus P;C;\sigma \Rightarrow  \bot\text{ if } m\neq n\) \\
		\vspace{0.3cm}
		\textcolor[rgb]{0.55,0,0}{(Occ)} \(\{x \simeq^? t\} \uplus P;C;\sigma \Rightarrow  \bot\text{ if there is an occurrence cycle of }x\text{ in }t\) \\
		\vspace{0.3cm}
		\textcolor[rgb]{0.55,0,0}{(VO)} \(\{x \simeq^? y, \overline{x_n\simeq^? y_n}\};C;\sigma \Rightarrow  \{\overline{x_n\simeq^? y_n}\}\{x\mapsto y\}; C;\sigma\{x\mapsto y\}\) \\
		\vspace{0.3cm}
	
  \end{frame}	
		
%---------------------------------------------------------------
		
		\begin{frame}[fragile=singleslide]
	\frametitle{Rules for Neighbourhood Constraints}
		\textcolor[rgb]{0.55,0,0}{(FFS)} $\{f \approx^? g\} \uplus C; \Phi \Rightarrow  C; \Phi; \text{ if } \mathcal{R}(f,g)\geq \lambda$ \\
		\vspace{0.3cm}
		\textcolor[rgb]{0.55,0,0}{(NFS)} $\{N \approx^? g\} \uplus C; \Phi \Rightarrow  C; update(\Phi,N\rightarrow \textbf{pc}(g,\mathcal{R},\lambda))$  \\
		\vspace{0.3cm}
		\textcolor[rgb]{0.55,0,0}{(FSN)} $\{g \approx^? N\} \uplus C; \Phi \Rightarrow  \{N \approx^? g\} \cup C;\Phi$  \\
		\vspace{0.3cm}
		\textcolor[rgb]{0.55,0,0}{(NN1)} $\{N \approx^? M\} \uplus C; \Phi \Rightarrow  C; update(\Phi,N\rightarrow {f}, M\rightarrow \textbf{pc}(f,\mathcal{R},\lambda))$, where $N \in dom(\Phi )$, $f\in \Phi(N)$ \\
		\vspace{0.3cm}
		\textcolor[rgb]{0.55,0,0}{(NN2)} $\{M \approx^? N\} \uplus C; \Phi \Rightarrow  \{N \approx^? M\} \cup C;\Phi$, where $M \notin dom(\Phi )$, $N\in dom(\Phi)$ \\
		\vspace{0.3cm}
		\textcolor[rgb]{0.55,0,0}{(Fail1)} $\{f \approx^? g\} \uplus C; \Phi \Rightarrow  \perp$, if $\mathcal{R}(f,g) < \lambda$ \\
		\vspace{0.3cm}
		\textcolor[rgb]{0.55,0,0}{(Fail2)} $C; \Phi \Rightarrow  \perp$, if there exists $N \in dom(\Phi )$ such that $\Phi (N)=\emptyset$ \\
		\vspace{0.3cm}
	
  \end{frame}	
	
%---------------------------------------------------------------
		
		\begin{frame}[fragile=singleslide]
	\frametitle{Simple example}
	\begin{exampleblock}{Example - both success}
	\(p(x,z) =^? q(f(b),f(x))\) and \(R=\{(b,c'),(c',c),(p,q)\}\)\\
	\end{exampleblock}
	\textbf{Pre Unification} \\
	
	\(P=\{p \simeq^? q\}\)\\
	\(C=\{\}\)\\
	\(\sigma=\{\}\)\\
	\(\Rightarrow^{\text{Dec}}\)\\
	\(P=\{x \simeq^? f, z \simeq^? f\}\)\\
	\(C=\{p \approx^? q\}\)\\
	\(\sigma=\{\}\)\\
	\(\Rightarrow^{\text{VE}}\)\\
	%\(P=\{N_1(N_2) \simeq^? f(b), z \simeq^? f(N_1(N_2))\}\)\\
	%\(C=\{p \approx^? q\}\)\\
	%\(\sigma=\{x \mapsto N_1(N_2)\}\)\\
	%\(\Rightarrow^{\text{Dec}^2}\)\\
	
	\end{frame}
	\begin{frame}[fragile=singleslide]
	\frametitle{Simple example cont.}
	\(P=\{N_1(N_2) \simeq^? f(b), z \simeq^? f(N_1(N_2))\}\)\\
	\(C=\{p \approx^? q\}\)\\
	\(\sigma=\{x \mapsto N_1(N_2)\}\)\\
	\(\Rightarrow^{\text{Dec}^2}\)\\
	\(p(x,z) =^? q(f(b),f(x)) ; R=\{(a,a'),(a',b),(b,c'),(c',c),(p,q)\}\)\\
	\(P=\{z \simeq^? f\}\)\\
	\(C=\{p \approx^? q, N_1 \approx^? f, N_2 \approx^? b\}\)\\
	\(\sigma=\{x \mapsto N_1(N_2)\}\)\\
	\(\Rightarrow^{\text{VE}}\)\\
	\(P=\{N_3(N_4(N_5)) \simeq^? f(N_1(N_2))\}\)\\
	\(C=\{p \approx^? q, N_1 \approx^? f, N_2 \approx^? b\}\)\\
	\(\sigma=\{x \mapsto N_1(N_2),z \mapsto N_3(N_4(N_5))\}\)\\
	\(\Rightarrow^{\text{Dec}^3}\)\\
	\end{frame}
	
	\begin{frame}[fragile=singleslide]
	\frametitle{Simple example cont.}
	\(P=\{\}\)\\
	\(C=\{p \approx^? q, N_1 \approx^? f, N_2 \approx^? b, N_3 \approx^? f, N_4 \approx^? N_1, N_5 \approx^? N_2\}\)\\
	\(\sigma=\{x \mapsto N_1(N_2), z \mapsto N_3(N_4(N_5))\}\)\\
	
	\vspace{0.5cm}
	\textbf{Constraint Simplification}\\
	\(C=\{p \approx^? q, N_1 \approx^? f, N_2 \approx^? b, N_3 \approx^? f, N_4 \approx^? N_1, N_5 \approx^? N_2\}\)\\
	\(\Phi=\{\}\)\\
	\(\Rightarrow^{\text{FFS}}\)\\
	\(C=\{N_1 \approx^? f, N_2 \approx^? b, N_3 \approx^? f, N_4 \approx^? N_1, N_5 \approx^? N_2\}\)\\
	\(\Phi=\{\}\)\\
	\(\Rightarrow^{\text{NFS}^3}\)\\
		\end{frame}
		
		\begin{frame}[fragile=singleslide]
	\frametitle{Simple example cont.}
	
	\textcolor[rgb]{0,0.58,0}{\begin{flushright}
	\(R=\{(b,c'),(c',c),(p,q)\}\)
	\end{flushright}}
	\(C=\{N_4 \approx^? N_1, N_5 \approx^? N_2\}\)\\
	\(\Phi=\{N1\mapsto \{f\}, N2\mapsto \{b,c'\}, N3\mapsto \{f\} \}\)\\
	\(\Rightarrow^{\text{NN2}}\)\\
	\(C=\{N_1 \approx^? N_4, N_5 \approx^? N_2\}\)\\
	\(\Phi=\{N1\mapsto \{f\}, N2\mapsto \{b,c'\}, N3\mapsto \{f\} \}\)\\
	\(\Rightarrow^{\text{NN1}}\)\\
	\(C=\{N_5 \approx^? N_2\}\)\\
	\(\Phi=\{N1\mapsto \{f\}, N2\mapsto \{b,c'\}, N3\mapsto \{f\} , N4\mapsto \{f\}\}\)\\
	\(\Rightarrow^{\text{NN2,NN1}}\)\\
		\end{frame}
		
		\begin{frame}[fragile=singleslide]
	\frametitle{Simple example cont.}
	\(C=\{\}\)\\
	\(\Phi_1=\{N1\mapsto \{f\}, N2\mapsto \{b\}, N3\mapsto \{f\} , N4\mapsto \{f\},N5\mapsto \{b,c'\}\}\)\\
	\(\Phi_2=\{N1\mapsto \{f\}, N2\mapsto \{c'\}, N3\mapsto \{f\} , N4\mapsto \{f\},N5\mapsto \{b,c',c\}\}\)\\
	
	\vspace{0.5cm}
	\textbf{Solution}\\
	\(\Phi_1=\{N1\mapsto \{f\}, N2\mapsto \{b\}, N3\mapsto \{f\} , N4\mapsto \{f\},N5\mapsto \{b,c'\}\}\)\\
	\(\Phi_2=\{N1\mapsto \{f\}, N2\mapsto \{c'\}, N3\mapsto \{f\} , N4\mapsto \{f\},N5\mapsto \{b,c',c\}\}\)\\
	\(\sigma=\{x \mapsto N_1(N_2), z \mapsto N_3(N_4(N_5))\}\)\\
		\end{frame}
		%---------------------------------------------------------------
		
	\begin{frame}[fragile=singleslide]
	\frametitle{Other examples}
	Simple examples where the pre-unification fails:
	\begin{itemize}
		\item [(Occ)] \(p(x) =^? q(f(x))\)
		\item [(Cla)] \(p(a,b) =^? q(f(x))\)
	\end{itemize}
	
		\vspace{0.5cm}
		
	\begin{exampleblock}{Example - PU success - CS fail}
	\(p(x,y,x) =^? q(f(a),g(b),y) ; R=\{(b,c),(c,d),(f,g),(p,q)\}\)\\
	\end{exampleblock}
	\(P=\{p \simeq^? q\}\)\\
		
		\textbf{Test 4 in Test.java}\\
		Pre - ok, CS - fail: \textcolor[rgb]{0,0.58,0}{- des ging bei mir, i hätt 5 Schritte und 2 Branches, die aber failen-}$p(x,y,x) =^? q(f(a),g(b),y) ; R=\{(b,c),(c,d),(f,g),(p,q)\}$\\
		\vspace{0.5cm}

		
		Solution: \ldots
	
  \end{frame}	
	
