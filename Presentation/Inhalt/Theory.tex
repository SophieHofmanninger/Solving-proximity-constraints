\section{Introduction} \subsection{}

\begin{frame}[fragile=singleslide]
	\frametitle{Motivation}
		For proving theorems, a frequently occurring problem is to find common instances of formulae.
		\begin{exampleblock}{Example 1}
			Let \(f\) be a function, \(a,b\) constants and \(x\) a variable. The two expressions
			\[f(a,x)\quad\text{and}\quad f(a,b)\]
			can be unified with \(\{x\mapsto b\}\).
		\end{exampleblock}

  \end{frame}	
		
%---------------------------------------------------------------

\begin{frame}[fragile=singleslide]
	\frametitle{Motivation}
		For proving theorems, a frequently occurring problem is to find common instances of formulae.
		\begin{exampleblock}{Example 2}
			Let \(f,g\) be functions, \(a,b\) constants and \(x\) a variable. The two expressions
			\[f(a,x)\quad\text{and}\quad g(a,b)\]
			cannot be unified as \(f\neq g\).
		\end{exampleblock}

  \end{frame}	
		
%---------------------------------------------------------------
				
\begin{frame}[fragile=singleslide]
	\frametitle{Motivation}
		In 1965 Robinson presented his unification algorithm and solved this problem, his algorithm was improved for better(=faster) performance since.\\ \ \\
		If we consider now the unification problem \[f(a,x)\simeq^{?} g(a,b)\] again, we might wonder, if we could not ignore \(f\neq g\), if they are ``close'' to each other, i.e. if they are equal in a fuzzy logic sense.\\
		Being close is represented as a proximity relation, which are symmetric and reflexive, but not necessarily transitive. C. Pau and T. Kutsia solved this problem, presenting an algorithm, which we implemented.

  \end{frame}	
		
%---------------------------------------------------------------


	\begin{frame}[fragile=singleslide]
	\frametitle{Introduction}
		4 sets:
		\begin{itemize}
			\item P: unification problem to be solved
			\item C: neighborhood constraint
			\item $\sigma$: set of pre-unifier
			\item $\psi$: name-class mapping
		\end{itemize}
  \end{frame}	
		
%---------------------------------------------------------------
		
	\begin{frame}[fragile=singleslide]
	\frametitle{Pre-Unification rules}
		(Tri) Trivial: \ldots \\
		(Dec) Decomposition: \ldots \\
		\ldots
	
  \end{frame}	
		
%---------------------------------------------------------------
		
		\begin{frame}[fragile=singleslide]
	\frametitle{Rules for Neigborhood Constraints}
		(FFS) Function Symbols: \ldots \\
		(NFS) Name vs Function Symbol: \ldots \\
		\ldots
	
  \end{frame}	
	
%---------------------------------------------------------------
		
		\begin{frame}[fragile=singleslide]
	\frametitle{Simple example}
		Problem to solve: $p(x,y) =? q(f(a),g(b))$\\
		\ \\
		Solution: \ldots
	
  \end{frame}	
	
%---------------------------------------------------------------

		\begin{frame}[fragile=singleslide]
	\frametitle{Steps Pre-Unification}
		\ldots
	
  \end{frame}	
	
%---------------------------------------------------------------

		\begin{frame}[fragile=singleslide]
	\frametitle{Steps Constraint-Simplification}
		\ldots
	
  \end{frame}	
	
%---------------------------------------------------------------
	