\section{Usage and Experience with the presented Tools} \subsection{}

\begin{frame}<presentation:0>[fragile=singleslide]
	\frametitle{Redmine/UML}
	We found Redmine and its features very helpful for Software development. Mainly, we used it to keep track of work that has to be done, that was done, and also see what the other person is working on. We also tried out the Forum, to communicate when we needed, but we found Whatsapp more suitable for this.\\
	For communication: In our opinion, communication is one of the most important things when working in a team, but we tried to restrict ourselves to the communications channels offered by the lecture, so basically Redmine, meeting a few minutes before the lecture, and Whatsapp that is like the Forum + notification when read.\\
	From a very early stage on, we also used UML, to express and communicate our ideas about the project and the implementation as well as to structure it.
  \end{frame}

%---------------------------------------------------------------

\begin{frame}[fragile=singleslide]
	\frametitle{Redmine/UML}
	\begin{itemize}
		\item Redmine - useful feature
		\item Communication:
			\begin{itemize}
				\item Redmine forum 
				\item Whatsapp
				\item meetings before the lectures
			\end{itemize}
		\item UML
		\begin{itemize}
				\item used it from the beginning
				\item to express and communicate our ideas
			\end{itemize}
	\end{itemize}
  \end{frame}
	
	
	%---------------------------------------------------------------
	\begin{frame}<presentation:0>[fragile=singleslide]
		\frametitle{Git/Javadoc}
	As it is a requirement for the project, we of course used git a lot. We tried to use branches and merge. As suggested, we committed (and most of the times also pushed) as soon as we further developed the project. As we did not work on the project for 8 hours straight every day in the week, this pretty much models the usage in a company, where you pull at the beginning, then work/commit to save your work and at the end of the day (or at a ``milestone'') push.\\
	It was very important to us to use Javadoc from the beginning on for almost each method. The reason for that was that eclipse has the nice feature of displaying it as a tooltip, and so you can understand easily the behaviour of the method, even if the other person implemented it.
  \end{frame}

%---------------------------------------------------------------
	\begin{frame}[fragile=singleslide]
		\frametitle{Git/Javadoc}
		\begin{itemize}
			\item Git
			\begin{itemize}
				\item own Git repository for the project 
				\item merged branches
				\item commited continuously
			\end{itemize}
			\item Javadoc
			\begin{itemize}
				\item used it from the beginning
				\item displaying it as a tooltip
			\end{itemize}
		\end{itemize}
  \end{frame}
	%---------------------------------------------------------------
	\begin{frame}<presentation:0>[fragile=singleslide]
			\frametitle{JUnit/Jenkins}
	For testing, that was very necessary, to find bugs, we used JUnit 5, although we later found out it is not too easy to use it from command line/Jenkins. It also was a good feeling, that whenever one had altered parts of the code, and run JUnit everything worked out again. At the beginning we had the feeling that Jenkins would not be as useful in our project as it might be in others, but actually, it was nice to use it to simply run all tests. Even though realizing that JUnit 5 has (also) strict naming conventions for test classes took us some time.
  \end{frame}
	
	%---------------------------------------------------------------
	\begin{frame}[fragile=singleslide]
			\frametitle{JUnit/Jenkins}
	\begin{itemize}
		\item JUnit 5
		\item good to find bugs
		\item not easy to call from the CMD/Jenkins
		\item JUnit 5 - strict naming of test classes
	\end{itemize}
  \end{frame}